% Appendix A

\chapter{Samples of Code} % Main appendix title

\label{AppendixA} % For referencing this appendix elsewhere, use \ref{AppendixA}

\lhead{Appendix A. \emph{lbfgsbnomessages.f90 from lbfgsbNS repository}} % This is for the header on each page - perhaps a shortened title

\lstset{language=[90]Fortran,
  basicstyle=\ttfamily\scriptsize,
  keywordstyle=\color{blue},
  commentstyle=\color{magenta},
  morecomment=[l]{!\ }% Comment only with space after !
  backgroundcolor=\color{white},
  numbers=left,
  numbersep=5pt,                   % how far the line-numbers are from the code
  breaklines=true,
  firstnumber = 2607
}
\section{Old Parts of the Code (Commented Out)}
All the code is available on \citep{lbfgsbNS} \label{ignoredcode}. This is also true for the code that is not being used anymore.  Here the old function dcsrch is commented out. lineww is replacing it instead
\begin{lstlisting}

!     call dcsrch(f,gd,stp,ftol,gtol,xtol,zero,stpmx,csave,isave,dsave)
      call lineww(f,gd,stp,ftol,gtol,xtol,zero,stpmx,csave,isave,dsave)

\end{lstlisting}

\lstset{
  escapeinside={\%*}{*)},
  firstnumber = 3687
}
\section{Stage 2 of the Line Search}
Stage 2 of the line search\citep{lbfgsbsoftware} \label{stage2}. This stage is not being run in lbfgsbNS. But is kept there for comparisson. For the new line search that replaces it \ref{linesearchww}
\begin{lstlisting}
      if (stage .eq. 1 .and. f .le. fx .and. f .gt. ftest) then

!  Here we define the modified function and derivative values.
!  This line search will only aim to satisfy the condition in (3.3) modified Armijo
         fm = f - stp*gtest
         fxm = fx - stx*gtest
         fym = fy - sty*gtest
         gm = g - gtest
         gxm = gx - gtest
         gym = gy - gtest

!  Call dcstep to update stx, sty, and to compute the new step.
!
         call dcstep(stx,fxm,gxm,sty,fym,gym,stp,fm,gm,brackt,stmin,stmax)

!        Reset the function and derivative values for f

         fx = fxm + stx*gtest
         fy = fym + sty*gtest
         gx = gxm + gtest
         gy = gym + gtest

      else
\end{lstlisting}

\section{Cubic and Quadratic line search sample}
This part of the code does not work in the case when the function is Non-Smooth \label{nsnowork}. For this reason it was eliminated from execution on lbfgsbNS and replaced with \ref{linesearchww}. stx in this case is a variable that represents the step with the minimum value so far.
\lstset{
  firstnumber = 3881
}

\begin{lstlisting}
!     First case: A higher function value. The minimum is bracketed. 
!     If the cubic step is closer to stx than the quadratic step, the 
!     cubic step is taken, otherwise the average of the cubic and 
!     quadratic steps is taken.
  ! theta, three, gamma are parameters
      if (fp .gt. fx) then
         theta = three*(fx - fp)/(stp - stx) + dx + dp
         s = max(abs(theta),abs(dx),abs(dp))
         gamma = s*sqrt((theta/s)**2 - (dx/s)*(dp/s))
         if (stp .lt. stx) gamma = -gamma
         p = (gamma - dx) + theta
         q = ((gamma - dx) + gamma) + dp
         r = p/q
         stpc = stx + r*(stp - stx) !quadratic step
         stpq = stx + ((dx/((fx - fp)/(stp - stx) + dx))/two)* &
                                                            (stp - stx) !The cubic step
         if (abs(stpc-stx) .lt. abs(stpq-stx)) then
            stpf = stpc
         else
            stpf = stpc + (stpq - stpc)/two
         endif
         brackt = .true.
\end{lstlisting}

\section{Line Search Enforcing Weak Wolfe Conditions}\label{linesearchww}
This is the implementation of the line search that enforces the weak Wolfe conditions. Bisections and expansions (as long as it doesn't mean leaving the bounding box) of the step length. This version tries to be as similar as possible to the interior of the while loop in qpspecial.m at hanso \citep{hanso}
\lstset{
  firstnumber = 4425
}

\begin{lstlisting}
      if (fp .ge. ftest) then 
         sty = stp
         fy = fp
         dy = dp
         brackt = .true.
      else
!     if second condition is violated not gone far enough (Wolfe)
         if (-dp .ge. gtol*(-ginit)) then
            stx = stp
            fx = fp
            dx = dp
         else
            return
         endif
      endif   
      
      !Bisection and expansion
      if (brackt) then
         stp = (sty + stx)/two
      else
         if (two * stp .le. stpmax) then !Remain within boundaries
            ! Still in expansion mode
            stp = two * stp
         else
            brackt = .true.
            sty = stp
            fy = fp
            dy = dp
         endif
      endif
      return
      end
\end{lstlisting}

\chapter{Automation}

This appendix includes some of the files that were used to run examples in parallel at the High Performance Clusters at NYU.  All of these files can be found in the repository \citep{lbfgsbNS}

\section{Script Generator} \label{pbsgenerator}

This is a sample of the script generator that creates pbs files to be sent to the High Performance Machines. The value ``i'' creates a different set of files, in this case one file for each value of $p$ that we want to test. To see a result of running this script look at \ref{pbsfile}. 
\lstset{
  language=sh,
  firstnumber = 1
}

\begin{lstlisting}
#!/bin/bash

for i in {0..20}
do
  cat > pbs.script.rosenbrock.$i <<EOF
#!/bin/bash

#PBS -l nodes=1:ppn=8,walltime=48:00:00
#PBS -m abe
#PBS -M weh227@nyu.edu
#PBS -N rosenbrockHD$((i))

module load gcc/4.7.3

cd /scratch/weh227/rosenbrock/
p=$(bc -l <<< "1+10 ^(-$((i)))")
for n in 1000 100000 1000000 10000000 100000000
do
   echo 10 \$n \$p
   ~/Documents/thesis/lbfgsfortran/./rosenbrockp 10 \$n \$p >> outputrosenbrock$((i)).txt
done

exit 0;

EOF
done

\end{lstlisting}
After they are created I could fire all of the runs at once by using the command 

\label{qsubfire}
for i in {0..20}; do qsub pbs.script.rosenbrock.$\$i$ ; done


\section{PBS File Sample} \label{pbsfile}
This is a sample file general with \ref{pbsgenerator}. It runs a special value $p$.  These are the ones that get started with the the qsub command.

Lines 4 to 5 provide the user with emails and critical information while running the process. Line 13 defines a run for different sizes of $n$. Line $16$ is the actual run that will be sent to an output file. In this case ``outputrosenbrock9.txt''
\lstset{
  language=sh,
  firstnumber = 1
}

\begin{lstlisting}

#!/bin/bash

#PBS -l nodes=1:ppn=8,walltime=48:00:00
#PBS -m abe
#PBS -M weh227@nyu.edu
#PBS -N rosenbrockHD9

module load gcc/4.7.3

cd /scratch/weh227/rosenbrock/
p=1.00000000100000000000
for n in 1000 100000 1000000 10000000 100000000
do
   echo 10 $n $p
   ~/Documents/thesis/lbfgsfortran/./rosenbrockp 10 $n $p >> outputrosenbrock9.txt
done

exit 0;



\end{lstlisting}
